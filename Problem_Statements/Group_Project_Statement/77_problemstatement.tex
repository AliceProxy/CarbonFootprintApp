\documentclass[10pt,draftclsnofoot,onecolumn,journal,compsoc]{IEEEtran}
\usepackage[margin=0.75in]{geometry}
\usepackage{graphicx}
\let\labelindent\relax
\usepackage{enumitem}
\usepackage{hyperref}
\usepackage{enumerate}
\usepackage[dvipsnames]{xcolor}
\usepackage{amssymb}
\usepackage{listings}

\begin{document}
\title{Project Statement - Ecological Footprint App}
\author{Sathya Ramanathan, Rohan Barve, Dominic Wasko \\
CS 461 | Fall 2018 \\
Oregon State University \\
}

\begin{titlepage}

    \pagenumbering{gobble}
    \centering
    \maketitle

\end{titlepage}

\pagenumbering{arabic}
\clearpage


\section{Project Abstract}

For our Senior Capstone project, we decided to work with the sustainability department here at Oregon State University to create a greener solution for the community. We have agreed with our client on the basis of creating an ecological footprint calculator specifically for the city of Corvallis. An ecological footprint is essentially a measurement of land area that is required to sustain a given population. Through this calculator, we can measure individually how much each person consumes the overall available land/water resources. We will be talking about the ecological problems we face today, the proposed solution, and our projects scope.


\section{Problem Description}

We live in a world where everything is finite. All resources are shared among us all. On top of that, we have rising environmental concerns such as air/water pollution. However, a great deal of this per say is caused by none other than humans. From the vehicles we drive to the food we eat, every action we take has a consequence on the environment in the long run. It is therefore important that we take care of our planet by making more sustainable choices. The underlying idea for our project is to promote a sustainable community through certain actions; and a great way to know one’s impact is by evaluating oneself. This awareness combined with the right choices can lead to a better planet for all.


\section{Problem Solution}

Our proposed solution is to create an ecological footprint calculator, so we can create awareness for the user. This
calculator will ask a set of questions that will then be used to formulate an individual footprint score. To do this, we
need to compare the users entered data with the national average footprint data. Ecological data is primarily collected by
offices in each city that monitors economic activities, agricultural productivity, energy consumption, and other related
information. This data can then be used for example to determine how much of a certain food was consumed, what kind
of wood is used to construct furniture, and how much energy a typical household uses. This data in return is compared
with the countrys average consumption profile to create an ecological footprint for the user. For example, the calculator
will ask the user how much meat they consume. If they answer none, this will greatly reduce their overall score. On the
other-hand, if they answered that they eat meat frequently, this will bump up their overall footprint score.


\section{Performance Metrics}

Alongside with the calculator, we are implementing a feature called local recommendations. This feature essentially
displays local services in Corvallis that can be used to reduce ones footprint. For example, someone who travels by car a
lot can look into transportation via bus or bike. This app would then provide helpful data such as displaying the nearest
bus stop or a local bike shop. Another feature we plan to have Amazon Alexa support. This is to create a choice for the
user whether to take the quiz through touch input or by voice. We believe this is a much more interesting way we can
interact with the user.   


\end{document}